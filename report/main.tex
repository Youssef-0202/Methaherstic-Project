\documentclass[12pt,a4paper]{article}

% --- Packages ---
\usepackage[utf8]{inputenc}
\usepackage[T1]{fontenc}
\usepackage[french]{babel} % Use [english] if preferred
\usepackage{amsmath, amssymb}
\usepackage{geometry}
\usepackage{graphicx}
\usepackage{hyperref}

\geometry{margin=1in}

% --- Title Information ---
\title{Projet de Métaheuristiques : Emploi du Temps Automatisé (FSTM)}
\author{Personne A : Spécialiste Données}
\date{\today}

\begin{document}

\maketitle

\tableofcontents
\newpage

\section{Introduction}
Ce rapport détaille la phase de prétraitement des données pour le problème de génération d'emplois du temps à la FSTM.

\section{Modélisation Mathématique}
Nous utilisons une fonction de coût pondérée $f(S)$ :
\begin{equation}
    f(S) = \sum_{i} w_H \cdot H_i(S) + \sum_{j} w_S \cdot S_j(S)
\end{equation}

\subsection{Contraintes Fortes (Hard Constraints)}
\begin{itemize}
    \item \textbf{H1 (Conflit Enseignant) :} Un enseignant ne peut pas donner deux cours simultanément.
    \item \textbf{H2 (Conflit Groupe) :} Un groupe d'étudiants ne peut pas assister à deux cours en même temps.
    \item \textbf{H3 (Conflit Salle) :} Une salle ne peut pas accueillir deux cours à la fois.
\end{itemize}

\section{Traitement des Données}
Les données ont été extraites du fichier Excel source et converties en fichiers CSV structurés :
\begin{itemize}
    \item \texttt{rooms.csv} : Liste des salles et capacités.
    \item \texttt{assignments.csv} : Sessions extraites (183 sessions).
    \item \texttt{groups.csv} : Cartographie des groupes par section (MST, CI, LST, TC).
\end{itemize}

\end{document}
